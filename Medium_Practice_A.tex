\documentclass{article}
\title{Medium Practice}
\date{15-11-2021}
\author{Aritra Roy}
\usepackage[margin=1in]{geometry}
\usepackage{esint}
\usepackage{amsmath}
\begin{document}
\maketitle
\begin{flalign}
1_A &= \left\{
    \begin{array}{cc}
        1 & \mbox{if } x \in A \\
        0 & \mbox{if } x \notin A
    \end{array}
\right.\\
n\underbrace{\uparrow\cdots\uparrow}n &=n\rightarrow n\rightarrow n
\end{flalign}
In the following, note the spacing between the = and the ${^1}1, {^2}2, and \,\, {^{{^3}3}} $
\begin{align}
1\uparrow1&={^1}1=1\\
2\uparrow\uparrow2&={^2}2=4
\end{align}
\[3\uparrow\uparrow\uparrow3={^{{^3}3}}3=3\uparrow\uparrow3\uparrow\uparrow3=\underbrace{3^{3^{3^{3^{3^{3^{.^{.^{.^3}}}}}}}}}_{3^{3^3} threes}\]
\begin{flalign}
\frac{d}{dx}f(x)&=\lim_{\Delta x\to 0}\frac{f(x+\Delta x)-f(x)}{\Delta x}\\
H_2O(l)+H_2O(l)&\rightleftharpoons H_3O^+(aq)+OH^-(aq)\\
\Gamma(n+1)\,&{\overset{def}{=}}\int_0^{\infty}e^{-t}t^ndt
\end{flalign}
\[gcd(n,\hspace{1mm}m\hspace{1mm}mod\hspace{1mm}n);\hspace{1em}x\equiv y\hspace{1em} (mod\hspace{1mm}b);\hspace{1em}x\equiv y\hspace{1em}mod\hspace{1mm}c;\hspace{1em}x\equiv y\]
In the following, note the bold symbols.
\begin{flalign}
\nabla\cdot\textbf{E} &=\frac{\rho}{\epsilon_0}\\
\nabla\cdot\textbf{B} &=0\\
\nabla\times\textbf{E} &=-\frac{\partial\textbf{B}}{\partial t}\\
\nabla\times\textbf{B} &=\mu_0\textbf{J}+\mu_0\epsilon_0\frac{\partial\textbf{E}}{\partial t}
\end{flalign}
For the following exercise, we need to use \textbf{\{esint\}} package to get the symbol of surface or volume integration.
\begin{flalign}
\oiint_{\partial V}\textbf{E}\cdot d\textbf{A}&=\frac{Q(V)}{\epsilon_0}\\
\oiint_{\partial V}\textbf{B}\cdot d\textbf{A}&=0\\
\oint_{\partial S}\textbf{E}\cdot d\textbf{l}&=-\frac{\partial\Phi_{B,S}}{\partial t}\\
\oint_{\partial S}\textbf{B}\cdot d\textbf{l}&=\mu_0I_S+\mu_0\epsilon_0\frac{\partial\Phi_{E,S}}{\partial t}
\end{flalign}
\end{document}