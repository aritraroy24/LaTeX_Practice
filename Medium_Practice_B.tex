\documentclass[10pt]{article}
\title{Medium Practice}
\date{15-11-2021}
\author{Aritra Roy}
\usepackage[margin=1in]{geometry}
\usepackage{amsmath, mathtools}
\begin{document}
\maketitle
One may find the environments \textbf{bmatrix} and \textbf{pmatrix} useful for the following exercises.
\[\rho_\theta=\begin{pmatrix}\cos\theta & \sin\theta\\-\sin\theta & \cos\theta \end{pmatrix}=\begin{bmatrix}\cos\theta & \sin\theta\\-\sin\theta & \cos\theta \end{bmatrix}\]
\[\left[\begin{tabular}{c|c c c}
    1&0&$\cdots$&0\\
    \hline\\
    0&*&$\cdots$&*\\
    $\vdots$&$\vdots$&$\ddots$&$\vdots$\\
    0&*&$\cdots$&*
\end{tabular}\right]=
\begin{tabular}{|c|c c c|}
    \hline
    1&0&$\cdots$&0\\
    \hline
    0&*&$\cdots$&*\\
    $\vdots$&$\vdots$&$\ddots$&$\vdots$\\
    0&*&$\cdots$&*\\
    \hline
\end{tabular}
\]
Note the locations of the bounds on the summation in the following exercise.
\[\sigma=\sqrt{\frac{1}{N}\sum_{i=1}^Np_i{(x_i-\overline{x})}^2}=\sqrt{\frac{\displaystyle\sum_{i=1}^Np_i{(x_i-\overline{x})}^2}{N}} \]
\begin{equation}
\varphi(n)=n\cdot\prod_{\begin{gathered}{p|n}\\{p\hspace{1mm}prime}\end{gathered}}\left(1-\frac{1}{p}\right)
\end{equation}
\end{document}